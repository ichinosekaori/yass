\documentclass{article}

\newcommand{\name}{YASS}

\usepackage{booktabs}
\usepackage{hyperref}
\usepackage{syntax}
\usepackage[skip=0.5em]{parskip}
\usepackage{amsmath}

\title{Design of \name}
\author{Ichinose Kaori}

\begin{document}

\maketitle

\section{Deviations from the R$^7$RS-small Standard}

\label{deviations}

\begin{itemize}
\item In section 4.1, inclusion is removed as a primitive expression type.
\item Macros are Common-Lisp-style.
\item In section 5.3, variable definitions are constrained to the first form.
\item In section 6.2, \name{} implements only operations on signed 61-bit integers and 64-bit
  floating point numbers, and conversions between them. Numeric operations are not
  polymorphic.
\item Datum labels are removed.
\item Libraries are removed.
\item \verb|Begin| is no longer used for splicing definitions into a context, except at
  the top-level.
\end{itemize}

\section{Core Grammar}

This section describes the core grammar of the language
implemented by \name.

\begin{grammar}

  <program> ::= <definition> <program>
  \alt empty

  <definition> ::= (define <identifier> <expr>)
  \alt (define-syntax <identifier> <expr>)

  <expr> ::= <identifier>
  \alt (quote <datum>)
  \alt (if <expr> <expr>)
  \alt (if <expr> <expr> <expr>)
  \alt (lambda <formals> <expr>)
  \alt <application>

  <application> ::= (<expr>)
  \alt <expr> :: <application>

  <formals> ::= <identifier>
  \alt ()
  \alt <identifier> :: <formals>
  
\end{grammar}

\section{Pass Organization}

\name{} is a micropass compiler with a similar structure to SML/NJ. Its hierarchy of IRs include:

\begin{description}
\item[Source language] Scheme with deviations from the standard described in section
  \ref{deviations}.
\item[Primitives] ** after macro expansion. Contains variable references, quotes,
  applications, abstractions, and conditionals.
\item[AST] ** in AST form so that calls to primitives generated in later passes would not
  shadow user program variables.
\item[Unique-names] ** where local names are unique.
\item[CPS] CPS with added primitives.
\item[Known-adic] ** where variadic functions are eliminated. This is done by converting
  functions so that they take a single list as an argument and deconstruct the list
  in the function body. \verb|Vector|, \verb|vector->list|, and \verb|vector-ref| are added as primitives.
\item[ClosurePS] ** where functions are closed. YASS uses flat closures.
  \verb|Make-closure|, \verb|call-closure|, and \verb|closure-ref| are added as primitives.
\item[VM] ** where names are replaced with indexed accesses. \verb|Argument-ref| is added
  as a primitive.
\item[Registered] ** where the number of arguments of functions are bounded.
\item[Machine language] Machine language. This is a very small subset of the
  target.
\end{description}

\section{Runtime Specifics}

The runtime runs on top of the C runtime. This enables \name{} programs
to call the operating system or to use C libraries through a C FFI, and
the runtime to be portable.

Garbage collection is done by checking for a GC cycle before each CPSed function.

\section{Datum Representation}

Data in YASS are represented as 64-bit
tagged pointers, with the 3 LSBs for the tag. The meanings of the tag bits
are shown in table \ref{tab:tags}.

\begin{table}
  \centering
  \begin{tabular}{cc}
    \toprule
    Tag bits & Meaning \\
    \midrule
    \verb|000| & 61-bit signed integer \\
    \verb|001| & character \\
    \verb|010| & symbol \\
    \verb|011| & special values (null, boolean, errors, etc.) \\
    \verb|100| & pointer to heap \\
    \verb|101| & unused \\
    \verb|110| & unused \\
    \verb|111| & unused \\
    \bottomrule
  \end{tabular}
  \caption{Meanings of tag bits}
  \label{tab:tags}
\end{table}

\begin{table}
  \centering
  \begin{tabular}{cc}
    \toprule
    Value & Meaning \\
    \midrule
    0 & false \\
    1 & true \\
    2 & null \\
    \bottomrule
  \end{tabular}
  \caption{Special values}
  \label{tab:special}
\end{table}

Objects on the heap are aligned to the 64-bit barrier and the formats
for storing them are as follows:

\begin{description}
\item[Pair] A pointer with tag $0$ at position $0$ and value $0$ and car and cdr
  follow.
\item[Port] A pointer with tag $1$ and value the \verb|FILE *| at
  position $0$.
\item[Procedure] A pointer with tag $0$ and value $1$ at position $0$ and a pointer to machine code and vector to enclosed values follows.
\item[String] A pointer with tag $2$ and value the length of the
  string at position $0$ and elements follow.
\item[Vector] Similar to strings but with a tag of $3$.
\item[Floating point number] A pointer with tag $0$ at position $0$ and value $2$
  and the floating point number representation follows.
\item[Reference cell] A pointer with tag $0$ at position $0$ and value $3$ and
  the boxed pointer follows.
\item[Tombstone] A pointer with tag $0$ at position $0$ and value $4$,
  and the forward pointer follows.
\end{description}

\end{document}
