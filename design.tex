\documentclass{article}

\usepackage{booktabs}
\usepackage{hyperref}

\title{Design of YASS}
\author{Ichinose Kaori}

\begin{document}

\maketitle

\section{Deviations from the R$^7$RS-small Standard}

\label{deviations}

\begin{itemize}
\item In section 4.1, inclusion is removed as a primitive expression type.
\item Macros are Common-Lisp-style.
\item In section 5.3, variable definitions are constrained to the first form.
\item In section 6.2, YASS implements only operations on signed 61-bit integers and 64-bit floating point
  numbers, and conversions between them.
\end{itemize}

\section{Pass Organization}

YASS is a micropass compiler with a similar structure to SML/NJ. Its hierarchy of IRs are:

\begin{description}
\item[Source language] Scheme with deviations from the standard described in section
  \ref{deviations}.
\item[Primitives] ** after macro expansion. Contains variable references, quotes,
  applications, abstractions, conditionals, and assignments.
\item[AST] ** in AST form so that calls to primitives generated in later passes would not
  shadow user program variables.
\item[Unique-names] ** where local names are unique.
\item[Known-adic] ** where there are no variadic functions. Abstractions no longer have a
  rest argument. \verb|Cons| is added as a primitive.
\item[Assign-less] ** where there are no assignments. \verb|Make-box|, \verb|box-ref|, and
  \verb|box-set!| are added as primitives.
\item[CPS] CPS with added primitives.
\item[ClosurePS] ** where functions are closed. YASS uses flat closures.
  \verb|Make-closure|, \verb|call-closure|, and \verb|closure-ref| are added as primitives.
\item[VM] ** where names are replaced with indexed accesses. \verb|Argument-ref| is added
  as a primitive.
\item[Registered] ** where the number of arguments of functions are bounded.
\item[Machine language] Machine language. This is usually a very small subset of the
  target.
\end{description}

\section{Runtime Specifics}

Runtime of YASS runs on top of the C runtime. This enables YASS programs
to call the operating system or to use C libraries through a C FFI, and
the runtime to be portable.

\section{Datum Representation}

Data in YASS are represented as 64-bit
tagged pointers, with the 3 MSBs for the tag. The meanings of the tag bits
are shown in table \ref{tab:tags}.

\begin{table}
  \centering
  \begin{tabular}{cc}
    \toprule
    Tag bits & Meaning \\
    \midrule
    \verb|000| & 61-bit signed integer \\
    \verb|001| & character \\
    \verb|010| & symbol \\
    \verb|011| & special values (null, boolean, errors, etc.) \\
    \verb|100| & pointer to heap \\
    \verb|101| & unused \\
    \verb|110| & unused \\
    \verb|111| & unused \\
    \bottomrule
  \end{tabular}
  \caption{Meanings of tag bits}
  \label{tab:tags}
\end{table}

Objects on the heap are aligned to the 64-bit barrier and the formats
for storing them are as follows:

\begin{description}
\item[Bytevector] A pointer with tag $0$ and value the length at
  position $0$ and elements aligned to an 8-byte barrier follow.
\item[Pair] A pointer with tag $1$ at position $0$ and car and cdr
  follow.
\item[Port] A pointer with tag $2$ and value the \verb|FILE *| at
  position $0$.
\item[Procedure] A pointer with tag $3$ and value the length of the
  procedure specification at position $0$ and the machine code follows.
\item[String] A pointer with tag $4$ and value the length of the
  string at position $0$ and elements follow.
\item[Vector] Similar to strings but with a tag of $5$.
\item[Floating point number] A pointer with tag $6$ at position $0$
  and the floating point number representation follows.
\item[Record] A pointer with tag $7$ and value type of the record,
  then a number representing the number of fields, then fields.
\item[Environment] A special record type. Contains an associative map
  of syntactic keywords to syntax transformers, and an associative map
  from top-level variable names to values.
\end{description}

\end{document}